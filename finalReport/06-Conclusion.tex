% !TEX root =./main.tex

\section{Conclusion}
We successfully implemented the first five stages of the sonar system. The final image produced matches the
 reference image provided with the project materials. We rearranged the stages somewhat by moving bias removal
 into stage 3 (band limiting and denoising) and rearranging the stages so that stage 3 came first, followed by
 stage 1, stage 2, stage 4, and stage 5. The purpose of that was to allow for a simpler, more reliable method
 of implementing the required functionality. Overall, we aimed for quality first, then speed, which is reflected
 in our manner of implementation in code. We used vector operations where possible and minimized unnecessary
 memory copies or needless recalculations.

The primary challenge we faced was, in fact, not even in the actual work in \textsc{Matlab}. Instead, a bigger source of
 issues was the Git version control system, which caused conflicts and even odd glitches in earlier drafts of
 this report. The lesson from that experience is that often times, it is not the actual engineering and building
 which causes issues, but all the things surrounding and supporting those.

\section*{Documentation}
We worked as a team on this project.  Additionally, we used resources such as the Mathworks website for \textsc{Matlab} syntax help (ex, confirming how to index every other entry of a matrix) and had a discussion with ChatGPT regarding recording phase, available \href{https://chatgpt.com/c/673ff615-bf94-800b-a437-6f861b769c24}{at this link}.  We also used that same ChatGPT conversation to discuss the mathematics behind accounting for time gain compensation.