% !TEX root =./main.tex

\section{Block 6: Quadrature Demodulation - Chen & Stentiford}

The quadrature demodulator is used to detect the pulse envelopes. It is a type of 
amplitude modulation (AM) decoder which is able to function without knowing the exact 
phase of the signal's carrier wave. It consists of three stages: mixing, filtering, 
and magnitude calculation.

The mixing stage of a normal (synchronous) demodulator multiplies the received signal 
by the original carrier wave, matching the frequency and phase. Because we do not know 
the phase in this case, we instead split the incoming signal, multipling each copy by 
carrier waves 180° apart. Or, in mathematical terms:

\begin{align*}
    y[n] = x[n]*(\cos(2 \pi f n)+j \cos(2 \pi f n+\pi)) = x[n]*(\cos(2 \pi f n)+j \sin(2 \pi f n))
\end{align*}

To mitigate the relative slowness of trigonometric calculations, we pre-computed a pair of 
tables for cosine and sine which could be used in the actual block instead of calls to 
MATLAB's cosine and sine functions. The tables additionally already account for the sampling 
rate and carrier frequency in order to minimize multiplication operations inside the time-
sensitive block.

After filtering, to merge the imaginary and real components back into a singal envelope, we 
simply take the magnitude:

\begin{align*}
    Y = \sqrt{I^2 + R^2}
\end{align*}

This is achieved as a single-line vector operation, and so is very performant.